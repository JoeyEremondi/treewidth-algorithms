\documentclass[handout]{beamer}
\usetheme{Warsaw}
\setbeamertemplate{footline}{}
\usepackage[french]{babel}
\usepackage[utf8]{inputenc}
\usepackage{amsmath,amsthm, amssymb, latexsym}
\boldmath
\usepackage[orientation=landscape,size=a4,scale=1.0]{beamerposter}

\newtheorem*{mdef}{Définition}
\newtheorem*{mthm}{Théorème}

\begin{document}

\begin{frame}

\begin{mdef}
On dit que $G$ est $triangul\acute{e}$ si pour tout cycle de longueur supérieure
strictement à 3, il existe une arête entre deux sommets non-adjacents
du cycle. La $largeur$ d'un graphe triangulé est la taille de sa plus
grande clique moins 1.
\end{mdef}

\begin{mdef}
La $largeur\ arborescente$ d'un graphe est la largeur minimale obtenable
parmi toutes les triangulations possibles.
\end{mdef}

\begin{mdef}
Un $dtree$ d'un graphe $G$ non-orienté est un arbre binaire
complet, dont les feuilles sont les arêtes de $G$.
\end{mdef}

Il permet de guider les algorithmes de partitionnement de type
\emph{divide-et-conquere}.

\end{frame}
\end{document}
