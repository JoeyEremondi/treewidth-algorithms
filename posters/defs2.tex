\documentclass[handout]{beamer}
\usetheme{Warsaw}
\setbeamertemplate{footline}{}
\usepackage[french]{babel}
\usepackage[utf8]{inputenc}
\usepackage{amsmath,amsthm, amssymb, latexsym}
\boldmath
\usepackage[orientation=landscape,size=a4,scale=3]{beamerposter}

\newtheorem*{mdef}{Définition}

\begin{document}

\begin{frame}

\begin{mdef}
Le $contexte$ d'un noeud $t$ d'un dtree est l'ensemble des noeuds
qui représente le bord du sous-graphe $G(t)$ par rapport au graphe
$G$. C'est-à-dire qu'un noeud $v$ apparait dans le contexte de $t$ si
et seulement s'il y a $v-u$ dans $G$ avec $v$ dans $G(t)$ et $u$ hors
de $G(t)$.
\end{mdef}

\begin{mdef}
Le $cutset$ d'un noeud $t$ d'un dtree est l'ensemble des noeuds
communs à $G(t_l)$ et à $G(t_r)$ et qui ne sont pas dans le contexte
de $t$.
\end{mdef}

\begin{mdef}
$$
cluster(t) = \left\{
    \begin{array}{ll}
        vars(t) & \mbox{ si $t$ est une feuille} \\
        cutset(t) \cup context(t) & \mbox{ sinon.}
    \end{array}
\right.
$$
\end{mdef}

Ceci permet de définir la largeur d'un $dtree$ : 
\begin{mdef}
La $largeur$ d'un dtree est la taille de son plus gros cluster
moins 1.
\end{mdef}

\end{frame}
\end{document}
