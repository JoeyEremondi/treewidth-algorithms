\documentclass[handout]{beamer}
\usetheme{Warsaw}
\setbeamertemplate{footline}{}
\usepackage[french]{babel}
\usepackage[utf8]{inputenc}
\usepackage{amsmath,amsthm, amssymb, latexsym}
\boldmath
\usepackage[orientation=landscape,size=a4,scale=3]{beamerposter}

\begin{document}

\newtheorem*{mdef}{Définition}
\newtheorem*{cor}{Corollaire}

\begin{frame}
\newtheorem*{prop_hd}{Proposition (Hopkins \& Derwiche)}
\begin{prop_hd}
   Soit un $dtree$ de largeur $w$ d'un graphe $G$,
   on peut construire une triangulation de largeur $w$ de $G$.
\end{prop_hd}
\begin{cor}
  S'il existe un $dtree$ de largeur $w$ d'un graphe $G$,
  alors la $largeur\ arborescente$ de $G$ est majorée par $w$.
\end{cor}

\begin{mdef}
On définit l'hypergraphe $H(G)$ du graphe $G$
$$H(G) = (E,\{ \{ (u,v) \in E \} \mid u \in V \})$$
\end{mdef}

\begin{block}{Construction d'un $dtree$}
  On créer un noeud du $dtree$ en partitionnant récursivement $H(G)$
  en deux sous hypergraphes associés aux fils du noeud.\\
  L'heuristique de partition se base sur l'équilibre des partitions.
\end{block}

\end{frame}
\end{document}
