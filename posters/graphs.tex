\documentclass[handout]{beamer}
%\usetheme{Warsaw}
\usepackage[french]{babel}
\usepackage[utf8]{inputenc}
\usepackage{amsmath,amsthm, amssymb, latexsym}
\usefonttheme[onlymath]{serif}
\boldmath
\usepackage[orientation=landscape,size=a4,scale=1.4]{beamerposter}
\usepackage[Large]{caption}
\usepackage{float}
\usepackage{subfig}
\usepackage{tikz}
\usetikzlibrary{topaths,calc}

\begin{document}
\begin{frame}
  \begin{block}
    \centering

\begin{figure}[H]
 \centering
\begin{tikzpicture}[auto, node distance=2cm]
  \node[] (1) {A};
  \node[below left of=1] (2) {B};
  \node[below right of=1] (3) {C};
  \node[below right of=2] (4) {D};
  \node[below right of=3] (5) {E};
 \path
  (1) edge node {} (2)
  (1) edge node {} (3)
  (2) edge node {} (3)
  (2) edge node {} (4)
  (3) edge node {} (4)
  (3) edge node {} (5);
\end{tikzpicture}
\caption{ Exemple de graphe $G$}
\end{figure}

\begin{figure}[H]
\centering 
%%%%%%%%% 1 %%%%%%%%%
\subfloat[$H(G)$]{ \fbox{
  \begin{tikzpicture}
    \node (ab) at (0,2) {};
    \node (ac) at (0,0) {};
    \node (bc) at (1,1) {};
    \node (bd) at (2,2.5) {};
    \node (cd) at (3.5,2) {};
    \node (ce) at (4,0) {};

    \begin{scope}[fill opacity=0.8]
    \filldraw[fill=green!80] ($(ac)+(-0.5,0)$)
        to[out=90,in=180] ($(bc)+(0,0.3)$)
        to[out=0,in=270] ($(cd)+(-0.5,-1)$)
        to[out=90,in=180] ($(cd)+(0,0.5)$)
        to[out=0,in=90] ($(cd)+(0.3,-0.5)$)
        to[out=270,in=90] ($(ce)+(0.3,-0.2)$)
        to[out=270,in=0] ($(ce)+(-2,0)$)
        to[out=180,in=270] ($(ac)+(-0.5,0)$);
    \filldraw[fill=yellow!70] ($(ab)+(-0.5,0)$) 
        to[out=90,in=180] ($(bd) + (0,0.5)$) 
        to[out=0,in=330] ($(bc) + (0,-0.5)$)
        to[out=150,in=300] ($(ab)+(-0.5,0)$);
    \filldraw[fill=blue!70] ($(ab)+(-0.5,0.2)$)
        to[out=90,in=180] ($(ab)+(0,0.5)$)
        to[out=0,in=130] ($(ac)+(0.5,0.3)$)
        to[out=310,in=0] ($(ac)+(0,-0.6)$)
        to[out=180,in=270] ($(ab)+(-0.5,0.2)$);
    \filldraw[fill=red!70] ($(bd)+(-0.5,0.2)$)
        to[out=90,in=163] ($(bd)+(0.3,0.5)$)
        to[out=340,in=90] ($(bc)+(0.5,-0.3)$)
        to[out=270,in=270] ($(bc)+(-0.5,-0.3)$)
        to[out=90,in=270] ($(bd)+(-0.5,0.2)$);
     \filldraw[fill=purple!70] ($(ce)+(-0.5,0.5)$)
        to[out=60,in=120] ($(ce)+(0.5,0.5)$)
        to[out=300,in=0] ($(ce)+(0,-0.5)$)
        to[out=180,in=240] ($(ce)+(-0.5,0.5)$);
    \end{scope}

    \fill (ab) circle (0.1) node [below] {AB};
    \fill (ac) circle (0.1) node [above] {AC};
    \fill (bc) circle (0.1) node [below] {BC};
    \fill (bd) circle (0.1) node [above] {BD};
    \fill (cd) circle (0.1) node [below] {CD};
    \fill (ce) circle (0.1) node [above] {CE};
  \end{tikzpicture}
} }
%%%%%%%%% 2 %%%%%%%%%
\subfloat[]{ \fbox{
  \begin{tikzpicture}
    \node (ab) at (0,2) {};
    \node (ac) at (0,0) {};
    \node (bc) at (1,1) {};
    \node (bd) at (2,2) {};
    \node (cd) at (3.5,2) {};
    \node (ce) at (4,0) {};

    \begin{scope}[fill opacity=0.8]
    \filldraw[fill=blue!70] ($(ab)+(-0.3,0.2)$)
        to[out=90,in=180] ($(ab)+(0,0.5)$)
        to[out=0,in=130] ($(ac)+(0.3,0.3)$)
        to[out=310,in=0] ($(ac)+(0,-0.6)$)
        to[out=180,in=270] ($(ab)+(-0.3,0.2)$);
    \filldraw[fill=red!70] ($(bd)+(-0.3,0.2)$)
        to[out=90,in=180] ($(bd)+(0.3,0.5)$)
        to[out=0,in=90] ($(bc)+(0.5,-0.3)$)
        to[out=270,in=270] ($(bc)+(-0.3,-0.3)$)
        to[out=90,in=270] ($(bd)+(-0.3,0.2)$);
     \filldraw[fill=purple!70] ($(ce)+(-0.5,0.5)$)
        to[out=60,in=120] ($(ce)+(0.5,0.5)$)
        to[out=300,in=0] ($(ce)+(0,-0.5)$)
        to[out=180,in=240] ($(ce)+(-0.5,0.5)$);
    \end{scope}
    
    \draw (1,3) node[anchor=west] {$1$} to[out=250,in=120] (1,-1);

    \fill (ab) circle (0.1) node [below] {AB};
    \fill (ac) circle (0.1) node [above] {AC};
    \fill (bc) circle (0.1) node [below] {BC};
    \fill (bd) circle (0.1) node [above] {BD};
    \fill (cd) circle (0.1) node [below] {CD};
    \fill (ce) circle (0.1) node [above] {CE};
  \end{tikzpicture}
} }
%%%%%%%%% 3 %%%%%%%%%
\subfloat[]{ \fbox{
  \begin{tikzpicture}
    \node (ab) at (0,2) {};
    \node (ac) at (0,0) {};
    \node (bc) at (1,1) {};
    \node (bd) at (2,2) {};
    \node (cd) at (3.5,2) {};
    \node (ce) at (4,0) {};

    \begin{scope}[fill opacity=0.8]
    \filldraw[fill=red!70] ($(bd)+(-0.3,0.2)$)
        to[out=90,in=180] ($(bd)+(0.3,0.5)$)
        to[out=0,in=90] ($(bc)+(0.5,-0.3)$)
        to[out=270,in=270] ($(bc)+(-0.3,-0.3)$)
        to[out=90,in=270] ($(bd)+(-0.3,0.2)$);
     \filldraw[fill=purple!70] ($(ce)+(-0.5,0.5)$)
        to[out=60,in=120] ($(ce)+(0.5,0.5)$)
        to[out=300,in=0] ($(ce)+(0,-0.5)$)
        to[out=180,in=240] ($(ce)+(-0.5,0.5)$);
    \end{scope}

    \draw (1,3) node[anchor=west] {$1$} to[out=250,in=120] (1,-1);
    \draw (-1,1) node[anchor=south] {$2$} to (0.5,1);
    \draw (5,1.5) node[anchor=south] {$3$} to[out=190,in=60] (2,-1);

    \fill (ab) circle (0.1) node [below] {AB};
    \fill (ac) circle (0.1) node [above] {AC};
    \fill (bc) circle (0.1) node [below] {BC};
    \fill (bd) circle (0.1) node [above] {BD};
    \fill (cd) circle (0.1) node [below] {CD};
    \fill (ce) circle (0.1) node [above] {CE};
  \end{tikzpicture}
} }
%%%%%%%%% 4 %%%%%%%%%
\subfloat[]{ \fbox{
  \begin{tikzpicture}
    \node (ab) at (0,2) {};
    \node (ac) at (0,0) {};
    \node (bc) at (1,1) {};
    \node (bd) at (2,2) {};
    \node (cd) at (3.5,2) {};
    \node (ce) at (4,0) {};

    \draw (1,3) node[anchor=west] {$1$} to[out=250,in=120] (1,-1);
    \draw (-1,1) node[anchor=south] {$2$} to (0.5,1);
    \draw (5,1.5) node[anchor=south] {$3$} to[out=190,in=60] (2,-1);
    \draw (0.7,2) node[anchor=south west] {$4$} to[out=340,in=120] (2.9,0.3);
    \draw (3.3,3) node[anchor=east] {$5$} to[out=240,in=70] (2.3,1.1);

    \fill (ab) circle (0.1) node [below] {AB};
    \fill (ac) circle (0.1) node [above] {AC};
    \fill (bc) circle (0.1) node [below] {BC};
    \fill (bd) circle (0.1) node [above] {BD};
    \fill (cd) circle (0.1) node [below] {CD};
    \fill (ce) circle (0.1) node [above] {CE};
  \end{tikzpicture}
} }
\caption{ Partitionnements successifs de l'hypergraphe $H(G)$}
\end{figure}


%%%%%%%%% TREE %%%%%%%%%
\begin{figure}[H]
  \tikzstyle{internal} = [rectangle, draw]
  \tikzstyle{level 1} = [sibling distance=4cm]
  \tikzstyle{level 2} = [sibling distance=3cm]
  \tikzstyle{level 3} = [sibling distance=2cm]
  \tikzstyle{level 4} = [sibling distance=1.5cm]
  \centering
 \begin{tikzpicture}
    \node [internal, label=$1$] (0) {(\{B,C\}, \{\})}
    child{
      node [internal, label=$2$] (1) { (\{A\}, \{B,C\}) }
      child{
        node[] (2) { AB }
      }
      child{
        node[] (3) { AC }
      }
    }
    child{ 
      node [internal, label=$3$] (4) { (\{\}, \{B,C\}) }
      child{
        node[] (5) { CE }
      }
      child{
        node [internal, label=$4$] (6) { (\{\}, \{B,C\}) }
        child{
          node [] (7) {BC}
        }
        child {
          node [internal, label=$5$] (8)  { (\{D\}, \{B,C\}) }
          child{
            node[] (9) {CD}
          }
          child{
            node[] (10) {BD}
          }
        }
      }
    };
 \end{tikzpicture}
\caption{dtree de taille 2=3-1 construit à partir du partitionnement, les contextes et cutsets sont affichés dans les noeuds internes}
\end{figure}

  \end{block}
\end{frame}
\end{document}
  
