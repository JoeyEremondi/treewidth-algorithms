\documentclass[handout]{beamer}
\usetheme{Warsaw}
\setbeamertemplate{footline}{}
\usepackage[french]{babel}
\usepackage[utf8]{inputenc}
\usepackage{amsmath,amsthm, amssymb, latexsym}
\usefonttheme[onlymath]{serif}
\boldmath
\usepackage[orientation=landscape,size=a4,scale=3]{beamerposter}

\title{A Practical Relaxation of Constant-factor Treewidth
        Approximation Algorithms}
\author{Émile Contal et Charles-Pierre Astolfi}
\date{MVA semestre 1}

\begin{document}
\begin{frame}
  \maketitle
  \begin{block}
    \centering
    Le but de notre travail a été d'évaluer un algorithme qui permet
    d'approximer la largeur arborescente (treewidth) d'un graphe.
    Beaucoup de problèmes de graphes comme l'inférence bayésienne peuvent
    être résolus en temps polynomial sur des graphes de largeur
    arborescente bornée. 
    \cite{bayesian_inferance} % http://www.cs.ru.nl/~johank/ECAI-623.pdf
    Le calcul exact de la largeur arborescente est un
    problème NP-complet et nécessite l'utilisation d'heuristiques.
  \end{block}
\end{frame}
\end{document}  
