\documentclass{article}

\usepackage[utf8]{inputenc}
%\usepackage{a4wide}
\usepackage{amsthm}
\usepackage{amsmath}
\usepackage{algorithm}

\title{Modèles graphiques probabilistes}
\author{Charles-Pierre Astolfi, Émile Contal} 
\date{3 janvier 2012}

\begin{document}
\newtheorem*{mdef}{Définition}
\newtheorem*{mthm}{Théorème}

\maketitle

\begin{abstract}
Le but de cet article de donner un algorithme qui permette
d'approximer la largeur arborescente (treewidth) d'un graphe.

Beaucoup de problèmes de graphes (par exemple, l'inférence bayesienne)
peuvent être résolus en temps polynomial sur des graphes de largeur
arborescente bornée. Le calcul exact de la largeur arborescente est un
problème est NP-complet et nécessite l'utilisation d'heuristiques. Ce
rapport est dédié à l'étude d'un de ces algorithmes, décrit dans
\cite{rootpaper}.

\end{abstract}

\section{Introduction}
Dans le cadre de l'article ``A Practical Relaxation of Constant-Factor
Treewidth Approximation Algorithms'' (\cite{rootpaper}), nous avons
implémenté l'algorithme proposé et l'avons comparé à d'autres
algorithmes de calcul de treewidth.


\section{Théorie}

Nous commençons cette section par quelques définitions qui nous seront
utiles dans la suite.

\begin{mdef}
Soit $G = (V,E)$ un graphe non-orienté sans boucle. On dit que $G$ est
connecté s'il existe un chemin entre chaque paire de sommets de $G$.
\end{mdef}

\begin{mdef}
On dit que $G$ est triangulé si pour tout cycle de longueur supérieure
strictement à 3, il existe une arête entre deux sommets non-adjacents
du cycle. La largeur d'un graphe triangulé est la taille de sa plus
grande clique moins 1.
\end{mdef}

\begin{mdef}
La largeur arborescente d'un graphe est la largeur minimale obtenable
parmi toutes les triangulations possibles.
\end{mdef}

\begin{mdef}
Un \emph{dtree} d'un graphe $G$ non-orienté est un arbre binaire
complet, dont les feuilles sont les arêtes de $G$.
\end{mdef}

Il permet de guider les algorithmes de partitionnement de type
\emph{divide-et-conquere}.




Il existe une définition de la largueur d'un dtree telle qu'on puisse
construire une triangulation du graphe $G$ dont la largeur
arborescente est la largeur du dtree, ce qui borne la largeur
arborescente de $G$.

Notre objectif est de donc construire un dtree de largeur la plus
petite possible ; pour ce faire on partitionne un hypergraphe $H(G)$
définit ainsi :
$$ H(G) = (E,\{ \{ (u,v) \in E \} \mid u \in V \}) $$

Le partitionnement de $H(G)$ est fait, comme dans l'article, grâce à
la librairie hMetis, qui utilise des heuristiques randomisées.

On sait qu'il existe des partitions qui mènent à un bon ratio
d'approximation de la largeur arborescente, mais bien qu'on ait pas de
garantie de trouver cette partition, l'algorithme semble battre en
pratique toutes les autres heuristiques connues.




\section{Implémentation}



\section{Résultats et discussion}


\bibliographystyle{amsplain}
\bibliography{biblio}

\end{document}
